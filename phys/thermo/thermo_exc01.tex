\documentclass[a4paper, 11pt]{ltjsarticle}
\usepackage{amsmath, mathtools, mathbbol, amssymb, bm, fancyhdr, anyfontsize, enumitem, graphicx, xcolor, ulem, setspace}

\pagestyle{fancy}
\lhead{熱力学01\ 確認問題}
\rhead{各10点}
\lfoot{学年・クラス・番号}
\cfoot{氏名}
\renewcommand{\footrulewidth}{0.4pt}

\begin{document}

\textbf{本日の演習}\quad 以下の問題を有効数字に注意して解け.
\begin{itemize}
  \item 名前が分かるように写真を撮り,授業時間内にLMSの所定の場所に提出すること.
  \item 解答例は授業終了後に,LMSから配信する.
\end{itemize}

\begin{enumerate}[label=問題. ]
  \item 冷蔵庫から出したジャムビンのふたが固くて開かない.そこでふたの部分だけを$60\ [^\circ\mathrm{C}]$のお湯に$30\ [\mathrm{s}]$ほど浸けたところ,簡単に開けられるようになった.ふたは鋼製,ビンはガラス製とする.
  \begin{enumerate}[label=\roman*)]
    \item なぜお湯に浸けるとふたが開きやすくなるのか,\textbf{熱膨張の観点から}説明せよ.\\[5pt]
    \vspace{125pt}
    \item 鋼の線膨張率を$1.2 \times 10^{-5}\ [/^\circ\mathrm{C}]$,初めの温度を$5.0\ [^\circ\mathrm{C}]$とする.ふたの直径が$6.0\ [\mathrm{cm}]$のとき,$60\ [^\circ\mathrm{C}]$に温めると\textbf{直径は何$[\mathrm{mm}]$増加するか}答えよ.\\[5pt]
    \vspace{125pt}
    \item ガラスの線膨張率は$0.9 \times 10^{-5}\ [/^\circ\mathrm{C}]$である.鋼製のふたとガラス製のビンを同じ温度,時間で温めた場合,どちらがより大きく膨張するか.\textbf{膨張後の直径を比較して}答えよ.
    \vspace{125pt}
  \end{enumerate}
\end{enumerate}

\end{document}
