\documentclass[a4paper, 11pt]{ltjsarticle}
\usepackage{amsmath, mathtools, mathbbol, amssymb, bm, fancyhdr, anyfontsize, enumitem, graphicx, xcolor, ulem, setspace}

\pagestyle{fancy}
\lhead{熱力学01\ 確認問題\textbf{\color{blue}「解答例」}}
\rhead{各10点}
\lfoot{学年・クラス・番号}
\cfoot{氏名}
\renewcommand{\footrulewidth}{0.4pt}

\begin{document}

\textbf{本日の演習}\quad 以下の問題を有効数字に注意して解け.
\begin{itemize}
  \item 名前が分かるように写真を撮り,授業時間内にLMSの所定の場所に提出すること.
  \item 解答例は授業終了後に,LMSから配信する.
\end{itemize}

\begin{enumerate}[label=問題. ]
  \item 冷蔵庫から出したジャムビンのふたが固くて開かない.そこでふたの部分だけを$60\ [^\circ\mathrm{C}]$のお湯に$30\ [\mathrm{s}]$ほど浸けたところ,簡単に開けられるようになった.ふたは鋼製,ビンはガラス製とする.
  \begin{enumerate}[label=\roman*)]
    \item なぜお湯に浸けるとふたが開きやすくなるのか,\textbf{熱膨張の観点から}説明せよ.\\[5pt]
    \textbf{\color{blue}
     お湯に浸けると,ビンのふたが温められて膨張し,ふたの直径が大きくなる.その結果,ビンの口に対してふたが締め付けられる力が弱まり,ふたが開けやすくなる.
    }\\[5pt]
    \item  鋼の線膨張率を$1.2 \times 10^{-5}\ [/^\circ\mathrm{C}]$,初めの温度を$5.0\ [^\circ\mathrm{C}]$とする.ふたの直径が$6.0\ [\mathrm{cm}]$のとき,$60\ [^\circ\mathrm{C}]$に温めると\textbf{直径は何$[\mathrm{mm}]$増加するか}答えよ.
    \\[5pt]
    \textbf{\color{blue}
     線膨張の式は次である.
     \begin{equation}
       \Delta D = D_0 \alpha \Delta t
     \end{equation}
      題意より,$D_0$,$\alpha$,$\Delta t$は以下である.
      \begin{itemize}
        \item $D_0 = 6.0\ [\mathrm{cm}] = 60\ [\mathrm{mm}]$
        \item $\alpha = 1.2 \times 10^{-5}\ [/^\circ\mathrm{C}]$
        \item $\Delta t = 60 - 5 = 55\ [^\circ\mathrm{C}]$
      \end{itemize}
      \vspace{5pt}
       $(1)$式に代入すると次である.
      \begin{equation}
        \Delta D = 60 \times 1.2 \times 10^{-5} \times 55 = 3,960 \times 10^{-5}
      \end{equation}
      \begin{flushright}
        \underline{\therefore\ 約$4.0 \times 10^{-2}\ [\mathrm{mm}]$増加する.}
      \end{flushright}
    }
    \vspace{25pt}
    \item ガラスの線膨張率は$0.9 \times 10^{-5}\ [/^\circ\mathrm{C}]$である.鋼製のふたとガラス製のビンを同じ温度,時間で温めた場合,どちらがより大きく膨張するか.\textbf{膨張後の直径を比較して}答えよ.
    \\[5pt]
    \textbf{\color{blue}
     線膨張率を比較すると以下である.
      \begin{itemize}
        \item 鋼製のふた: $\alpha_{\mathrm{steal}} = 1.2 \times 10^{-5}\ [/^\circ\mathrm{C}]$
        \item ガラス製のビン: $\alpha_{\mathrm{glass}} = 0.9 \times 10^{-5}\ [/^\circ\mathrm{C}]$
      \end{itemize}
       鋼製は$\mathrm{ii})$で求めたので$\alpha_{\mathrm{glass}}$を$(1)$式に代入すると次である.
      \begin{equation}
        \Delta D_{\mathrm{glass}} = 60 \times 0.9 \times 10^{-5} \times 55 = 2,970 \times 10^{-5}
      \end{equation}
      \begin{flushright}
        \underline{\therefore\ $4.0 \times 10^{-2} < 3.0 \times 10^{-2}$より,鋼製のふたの方がより大きく膨張する.}
      \end{flushright}
    }
  \end{enumerate}
\end{enumerate}

\end{document}
