\documentclass[11pt]{beamer}
\usetheme{Singapore}
\usepackage{luatexja}
\usepackage{luatexja-fontspec}
\usepackage{unicode-math}
\usepackage{amsmath, mathtools, amssymb, fancyhdr, anyfontsize, subcaption, multirow, wrapfig, graphicx, url, enumitem, ascmac, tikz, tikz-3dplot}
\usepackage[numbers]{natbib}
\usetikzlibrary{arrows, angles, quotes}
\setmathfont{Latin Modern Math}

\newcommand{\dlim}[2]{\displaystyle{\lim_{#1 \to #2}\:}}
\newcommand{\dint}[2]{\displaystyle{\int_{#1}^{#2}}}
\renewcommand{\cite}[1]{\textsuperscript{\citep{#1}}}
\def\tick#1#2{\draw[thick] (#1)++(#2:0.12) --++ (#2-180:0.24)}
\def\N{100}

\begin{document}

% main text
\begin{frame}[containsverbatim]
\frametitle{\textbf{等速直線運動}とは?}

\begin{description}
  \item[慣性の法則:] 物体が同じ運動をし続けようとする原則.
  \item[  $\Downarrow$]
  \item[等速直線運動:] 外から抵抗する力がはたらかないとき,物体に
  \item[       ] 力を加えると常に同じ速さで移動する運動.
  \item[]
  \item[速さ$v$:] 単位時間$t\ [\mathrm{s}]$あたりに移動した距離$x\ [\mathrm{m}]$.
\end{description}

\begin{equation}
  \dfrac{x\ [\mathrm{m}]}{t\ [\mathrm{s}]} = v\ [\mathrm{m/s}]
\end{equation}

\end{frame}

\begin{frame}[containsverbatim]
\frametitle{等速直線運動は\textbf{比例}の具体例である!}

比例の式は,

\begin{equation}
  y = ax\ (a = \mathrm{Const.})
\end{equation}

であった.

速さは(1)式を変形すると,

\begin{equation}
  x = vt
\end{equation}

\textbf{等速}直線運動なので式は,

\begin{equation}
  x = vt\ (v = \mathrm{Const.})
\end{equation}

(4)式は$x$と$t$が比例していると言えるので,(2)式と比較すると,

\begin{center}
  \colorbox{yellow}{$y$は$x$,$a$は$v$,$x$は$t$と対応している.}
\end{center}

\end{frame}

\begin{frame}[containsverbatim]
\frametitle{実験}

\begin{description}
  \item[1. ] レールの上に\ i)ドライアイスを乗せ,力を加える.
  \item[2. ] ストップウォッチを用いて10 cm置きの地点を通過したらラップする.
  \item[3. ] 横軸を時間$t$,縦軸を距離$x$($x-t$\ graph)とし,グラフ用紙にプロットする.
  \item[4. ] 1. ~ 3. を\ ii)石ケンを用いて繰り返す.
  \item[5. ] i)とii)のグラフを比較し,考察する.
\end{description}

\end{frame}

\end{document}