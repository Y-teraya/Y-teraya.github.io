\documentclass[a4paper,11pt]{ltjsarticle}
\usepackage{amsmath, mathtools, mathbbol, amssymb, bm, fancyhdr, anyfontsize, subcaption, multirow, wrapfig, graphicx, hyperref, url, enumitem, ascmac, tikz, tikz-3dplot, makeidx}
\usetikzlibrary{arrows, angles, quotes}
\captionsetup{compatibility=false}

\hypersetup{
 pdfencoding=auto,
 setpagesize=false,
 bookmarksnumbered=true,
 bookmarksopen=true,
 colorlinks=true,
 linkcolor=blue,
 citecolor=blue,
 urlcolor=blue,
}

\makeindex
\usepackage[numbers]{natbib}
\usetikzlibrary{arrows, angles, quotes}
\renewcommand{\cite}[1]{\textsuperscript{\citep{#1}}}
\newcommand{\idxitem}[2]{\item[#1\index{#1}] #2}

\title{\textbf{植物関連知識\ 内容まとめ}}
\author{Y-teraya}
\date{\today}

\begin{document}

\pagestyle{fancy}
\rhead{\textbf{\thepage}}
\cfoot{}

\maketitle

\begin{abstract}
  植物に関する知識を私個人および自主ゼミで学習した内容のまとめです.
  正確性には欠けますが,参考程度にご活用ください.
\end{abstract}

\tableofcontents

\vspace{12pt}

\begin{center}
  \textbf{\color{blue}青文字}をクリックすると,対応した箇所へ遷移します.
\end{center}

\clearpage

\renewcommand{\headrulewidth}{0pt}

\part[農業基礎]{農業基礎\cite{Agriculture:538}}

\section{人間と植物・作物との関わり\ pp. 16-19}

\begin{description}
  \idxitem{Word1}{Content}
\end{description}

\clearpage

\lhead{農業基礎}
\renewcommand{\headrulewidth}{0.4pt}
\renewcommand{\footrulewidth}{0.4pt}

\begin{description}
  \idxitem{Word2}{Content}
\end{description}

\clearpage

\lhead{}
\renewcommand{\headrulewidth}{0pt}
\renewcommand{\footrulewidth}{0pt}

\part[作物]{作物\cite{Crops:503}}

\section{作物とは何か\ pp. 8-16}

\begin{description}
  \idxitem{Word3}{Content}
\end{description}

\clearpage

\lhead{作物}
\renewcommand{\headrulewidth}{0.4pt}
\renewcommand{\footrulewidth}{0.4pt}

\begin{description}
  \idxitem{Word4}{Content}
\end{description}

\clearpage

\pagestyle{fancy}
\lhead{}
\rhead{\textbf{\thepage}}
\cfoot{}
\renewcommand{\headrulewidth}{0pt}
\renewcommand{\footrulewidth}{0pt}

\bibliographystyle{plain}
\bibliography{ref.bib}
\printindex

\end{document}