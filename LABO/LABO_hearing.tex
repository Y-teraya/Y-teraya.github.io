\documentclass[a4paper,11pt]{ltjsarticle}
\usepackage{fancyhdr}
\usepackage{ascmac}
\usepackage{tabularx}

\begin{document}

\pagestyle{fancy}
\lhead{植物ストレス生理学研究室 ヒアリング用紙}
\rhead{学生番号\ 23113236 \quad 名前\ 寺谷優輝}
\lfoot{}
\cfoot{\thepage}

\section{今考えている希望進路を教えてください.}

\begin{itemize}
    \item 就職希望 $\rightarrow$ 希望業界(分野)$\rightarrow$ 希望職種
    \item \textbf{大学院} $\rightarrow$ \textbf{内部進学} or 外部進学
\end{itemize}

\subsection{また,どの程度まで自分で進めていますか?}

できる限り推薦を取って,安定的に院試に挑みたいと考えている.そのため,まずは講義
に集中して高いGPAを取ることを最優先にしている.ただ,厳しい側面もあるのでバイオ技
術者認定試験(上級)を受験した後に,院試の対策を行う予定である.

\section{研究室で何を身につけたいと考えていますか.}

研究室では,計画性や他者との関わりを身に着けたいと考えている.まず,計画性は実験
を行うとき先を見通すことはするが,先を見通しすぎて\textbf{先のことを優先してしまう}傾向が
ある.また,自身の気が乗らないことは基本的にやらないという選択肢を取ることが多い.
そのため全体としては計画性,特に\textbf{優先順位付け}を1つ目に身に着けたいと考えている.

他者との関わりは,私の設計思想が尖っていることは分かっているが,\textbf{私1人ですべて行
ったほうが効率よく精度が良いものができる}と考えているので,他者を置き去りにして行
動してしまうことが多い.学生実験においては1人で行うことができたが,研究においては
様々な先行研究や知識があり協力して行うことが悪くいえば\textbf{強いられる状況下}になる.
このようなグループワークの苦手意識を減らせるように頑張りたい.


\section{教員が知っておいた方が良いことがありましたら,記入してください.}

特になし.

\section{不安なことなど何かありましたら,記入してください.}

同じ研究室に配属された方を考えると苦手としているタイプの方が複数いるため,
コミュニケーションに難が生じると考えている.

\end{document}