\documentclass[a4paper,11pt]{ltjsarticle}
\usepackage[margin=28truemm]{geometry}

\usepackage{booktabs}
\usepackage{amsmath, mathtools, mathbbol, amssymb, bm, cancel, fancyhdr, anyfontsize, subcaption, multirow, wrapfig, graphicx, url, enumitem, ascmac, tikz, tikz-3dplot, makeidx}
\usepackage[luatex]{hyperref}
\usepackage[most]{tcolorbox}
\usepackage{ascolorbox}
\usepackage{stackengine}
\usepackage{cleveref}
\usepackage{appendix, xcolor}
\usepackage{tablefootnote}
\usepackage{tabularx}

\newtcolorbox{marker}[1][]{enhanced,
  before skip=2mm,after skip=3mm,fontupper=\gtfamily\sffamily,
  boxrule=0.4pt,left=5mm,right=2mm,top=1mm,bottom=1mm,
  colback=yellow!50,
  colframe=yellow!20!black,
  sharp corners,rounded corners=southeast,arc is angular,arc=3mm,
  underlay={%
    \path[fill=tcbcolback!80!black] ([yshift=3mm]interior.south east)--++(-0.4,-0.1)--++(0.1,-0.2);
    \path[draw=tcbcolframe,shorten <=-0.05mm,shorten >=-0.05mm] ([yshift=3mm]interior.south east)--++(-0.4,-0.1)--++(0.1,-0.2);
    \path[fill=yellow!50!black,draw=none] (interior.south west) rectangle node[white]{\Huge\bfseries !} ([xshift=4mm]interior.north west);
    },
  drop fuzzy shadow,#1
  }

\begin{document}

\pagestyle{fancy}
\lhead{第10章: 非行問題の理解と対応(D,12/3)}
\rhead{}
\cfoot{\thepage}

\section{非行少年の分類と現状}

少年法第3条により,非行少年は以下の3つに分類される:

\begin{description}
    \item[犯罪少年] 14歳以上で犯罪を行った者
    \item[触法少年] 14歳未満で犯罪少年と同じ行為を行った者
    \item[ぐ犯少年] 今後犯罪少年や触法少年になる可能性の高い者
\end{description}

近年,非行少年数は減少傾向にあるが,\textbf{多様化・低年齢化}が進んでいる.

\begin{table}[htbp]
    \centering
    \caption{非行問題を理解する3つの危機}
    \begin{tabularx}{\textwidth}{XXXX}
        \hline
        危機の種類 &  & 内容 & 対応の特徴 \\
        \hline
        \textbf{発達的危機} & 遍在性 & 思春期・青年期特有の心理的不安定さ & 予防・開発的対応 \\
        \textbf{基本的危機} & 偏在性 & 精神障害や知的・発達障害を伴う場合 & 医療的理解と対応 \\
        \textbf{個人的危機} & 偏在性 & 困窮した経済状況,崩壊した家庭環境での生育 & 福祉的理解と対応 \\
        \hline
    \end{tabularx}
\end{table}

\begin{marker}
遍在性と偏在性を混同すると,過度な一般化が起こり,問題をより深刻化させる.
\end{marker}

\section{非行少年の3つの顕著な傾向}

\begin{description}
    \item[悪い自己イメージ] 劣等感・疎外感・低い自己肯定感
    \item[自己管理能力の低さ] 基本的生活習慣の欠如,金銭管理能力の不足
    \item[コミュニケーション能力の低さ] ソーシャルスキルの欠如,衝動や攻撃性のコントロール困難
\end{description}

\begin{ascolorbox9}{非行の悪化段階(ショウ・マッケイ,ベッカーの説)}[3]

    \begin{enumerate}
        \item 遊戯・最初の規則違反
        \item 社会的規範との葛藤(喫煙・飲酒・深夜徘徊)
        \item 非行行動の継続
        \item 非行集団への忠誠と一体感
        \item 組織的非行行動
        \item 非行ギャング化とヒエラルキー上昇
    \end{enumerate}

\end{ascolorbox9}

\section{ハーシの社会的絆理論(非行予防)}

非行行動は社会的絆の4つの強さによって抑制される:

\begin{description}
    \item[愛着] 両親・教師・仲間との情緒的つながり
    \item[投資] 教育や仕事への思い入れ・こだわり
    \item[巻き込み] 日常的な集団活動・余暇活動への参加
    \item[規範観念] 社会規範の受容と習慣化
\end{description}

現代の非行少年はこれらすべての要素が弱い傾向にある.

\section{学校における3段階の援助}

学校における対応は,N-of-1に考える必要もある:

\begin{table}[htbp]
    \centering
    \caption{非行問題を理解する3つの危機}
    \begin{tabularx}{\textwidth}{XXXX}
        \hline
        援助レベル &  & 対象 & 主な対応 \\
        \hline
        \textbf{1次的援助} & 発達支持的 & すべての子ども & 規範意識の学習,社会的絆の形成,構成的グループエンカウンター,ソーシャルスキル・トレーニング \\
        \textbf{2次的援助} & 課題未然防止・早期発見 & 配慮を要する子ども & 個別対応の継続,専門機関との連携,呼び出し面接(丁寧語で淡々と) \\
        \textbf{3次的援助} & 困難課題対応 & 犯罪を犯した少年 & 警察・家庭裁判所・児童相談所等の専門機関による処遇 \\
        \hline
    \end{tabularx}
\end{table}

\begin{ascolorbox4}{質問}

    \begin{enumerate}
        \item 最近の自己管理能力は,どうやって鍛えてますか.
        \item 社会的絆は,デジタル空間でも成立すると思いますか.
        \item 規範意識は,文化によって変わるものだと思いますか.
    \end{enumerate}

\end{ascolorbox4}

\end{document}