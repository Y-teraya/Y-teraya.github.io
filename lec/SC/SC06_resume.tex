\documentclass[a4paper,11pt]{ltjsarticle}
\usepackage[margin=28truemm]{geometry}
\makeatletter
\renewcommand{\section}{%
  \@startsection{section}{1}{\z@}%
    {1\Cvs \@plus.5\Cdp \@minus.2\Cdp}%
    {.5\Cvs \@plus.3\Cdp}%
    {\fontsize{14pt}{0pt}\selectfont\textsf}%
}
\renewcommand{\subsection}{%
  \@startsection{subsection}{2}{\z@}%
    {1\Cvs \@plus.3\Cdp \@minus.2\Cdp}%  % 前のスペース
    {.3\Cvs \@plus.2\Cdp}%                % 後のスペース
    {\fontsize{12pt}{0pt}\selectfont\textsf}%  % フォントサイズとスタイル
}
\makeatother
\usepackage{booktabs}
\usepackage{fancyhdr}
\usepackage{ascmac}
\usepackage{tabularx}

\begin{document}

\pagestyle{fancy}
\lhead{第6章: 進路指導とキャリア教育(D,11/5)}
\rhead{}
\cfoot{\thepage}

\section{職業指導からキャリア教育への変遷}

\begin{description}
    \item[職業指導] Parsons (1908)の職業選択支援運動が起源.日本では1927年に文部省通達,1947年の教育基本法で学校教育に導入.
    \item[進路指導] 1958年に「職業指導」から名称変更.就職希望者のみでなく,全生徒の成長・発達を意識した支援へ転換.1961年に文部省が「教師が組織的,継続的に援助する過程」と定義.
                    \textbf{キャリア教育の部分集合}である.
    \item[キャリア教育] 1999年の中央教育審議会答申で初登場.小学校段階から発達段階に応じた実施が提言され,2017年の学習指導要領で明記.
\end{description}

\section{キャリア教育の定義と目的}

\begin{description}
    \item[中央教育審議会(2011)の定義] 「一人一人の社会的・職業的自立に向け,必要な基盤となる能力や態度を育てることを通して,キャリア発達を促す教育」
\end{description}

\vspace{-10pt}

\renewcommand{\arraystretch}{1.5}
\begin{table}[htbp]
    \centering
    \caption{3つの意義}
    \begin{tabularx}{\textwidth}{|X|X|X|}
        \hline
        学校教育構成の理念と方向性の提示 & 各学校段階での発達課題の明確化 & 学校生活と社会生活・職業生活の連結による学習意欲の喚起 \\
        \hline
    \end{tabularx}
\end{table}

\section{基礎的・汎用的能力(4つの柱)}

\vspace{-10pt}

\renewcommand{\arraystretch}{1}
\begin{table}[htbp]
    \centering
    \caption{基礎的・汎用的能力(4つの柱)}
    \begin{tabularx}{\textwidth}{XXXX}
        \hline
        \textbf{人間関係形成・社会形成能力} & \textbf{自己理解・自己管理能力} & \textbf{課題対応能力} & \textbf{キャリアプランニング能力} \\
        \hline
        多様な他者理解,協力・協働,社会参画 & 肯定的自己理解,主体的行動,自己管理 & 課題発見・分析,計画立案,問題解決 & 「働くこと」の意義理解,主体的キャリア形成 \\
        \hline
    \end{tabularx}
\end{table}

※これら4つの能力は相互に関連・依存し,特定の順序や均一な習得を求めるものではない.

\section{キャリア教育の実践}

\vspace{-10pt}

\renewcommand{\arraystretch}{2}
\begin{table}[htbp]
    \centering
    \caption{進路指導の6つの活動(文部省,1995)}
    \begin{tabularx}{\textwidth}{|X|X|X|}
        \hline
        生徒理解と自己理解 & 進路情報の提供 & 啓発的経験の提供 \\
        \hline
        進路相談 & 就職・進学指導 & 卒業者の追指導 \\
        \hline
    \end{tabularx}
\end{table}

\clearpage

\subsection{特別活動との連携}
特別活動を「要」としつつ各教科等でキャリア教育を実施.「勤労生産・奉仕的行事」は体験的キャリア教育の重要機会.キャリア・パスポートを活用し,児童生徒の自己理解と将来設計を支援.

\section{評価と連携}

\vspace{-10pt}

\renewcommand{\arraystretch}{2}
\begin{table}[htbp]
    \centering
    \caption{3つの評価対象}
    \begin{tabularx}{\textwidth}{|X|X|X|}
        \hline
        生徒の学習状況 & 教師の学習指導 & 学校の指導計画 \\
        \hline
    \end{tabularx}
\end{table}

\begin{description}
    \item[評価方法] 観察,制作物,キャリア・パスポート,自己評価,相互評価,他者評価など.
    \item[連携の重要性] 外部連携(地域・社会との接続)と学校間連携(発達段階に応じた継続的支援)が不可欠.
\end{description}

\section{PDCAサイクルの活用}

\vspace{-10pt}

\renewcommand{\arraystretch}{1.8}
\begin{table}[htbp]
    \centering
    \caption{藤田(2014)が示す7つの重要ポイント}
    \begin{tabularx}{\textwidth}{|X|X|X|}
        \hline
        行動レベルでの現状把握と目標設定 & 教職員・保護者・地域の納得できる目標 & 基礎的・汎用的能力の活用 \\
        \hline
        評価指標の設定と成果評価 & 包括的評価の工夫 & 教科評価との区別 \\
        \hline
        評価結果に基づく改善 & & \\
        \hline
    \end{tabularx}
\end{table}

\begin{description}
    \item[A(改善)の重要性] 評価を踏まえた目標の再検討が特に重要.キャリア教育は継続的・体系的な実践が求められる.
\end{description}

\begin{boxnote}
    \begin{enumerate}
        \item 中学生時代の職場体験や職業調べにおいて,「このような仕事が存在するのか」と驚いた職業はありましたか.
        \item 「人間関係形成・社会形成能力」「自己理解・自己管理能力」「課題対応能力」「キャリアプランニング能力」の4つのうち,自身が最も得意または苦手とする能力はどれですか.
        \item 仮に自身が中学校教員として「勤労生産・奉仕的行事」を企画する場合,どのような活動を考えますか.
    \end{enumerate}
\end{boxnote}

\end{document}