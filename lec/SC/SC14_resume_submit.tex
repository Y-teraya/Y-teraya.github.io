\documentclass[a4paper,11pt]{ltjsarticle}
\usepackage[margin=28truemm]{geometry}
\makeatletter
\renewcommand{\section}{%
  \@startsection{section}{1}{\z@}%
    {1\Cvs \@plus.5\Cdp \@minus.2\Cdp}%
    {.5\Cvs \@plus.3\Cdp}%
    {\fontsize{14pt}{0pt}\selectfont\textsf}%
}
\renewcommand{\subsection}{%
  \@startsection{subsection}{2}{\z@}%
    {0.5\Cvs \@plus.5\Cdp \@minus.5\Cdp}%  % 前のスペース
    {.3\Cvs \@plus.2\Cdp}%                % 後のスペース
    {\fontsize{12pt}{0pt}\selectfont\textsf}%  % フォントサイズとスタイル
}
\makeatother
\usepackage{booktabs}
\usepackage{fancyhdr}
\usepackage{ascmac}
\usepackage{tabularx}
\usepackage{ascolorbox}

\begin{document}

\pagestyle{fancy}
\lhead{第14章: 進路指導・キャリア教育の展開(D,1/14)}
\rhead{BIO23113236\ 寺谷優輝}
\cfoot{\thepage}

\section{職場体験の意義と効果的な展開}

\subsection{職場体験とその意義}

\begin{description}
    \item[職場体験の定義] 「生徒が事業所などの職場で働くことを通じて,職業や仕事の実際について体験したり,働く人々と接したりする学習活動」である(文部科学省,2005).
\end{description}

文部科学省(2005)の「中学校職場体験ガイド」によると,職場体験の教育的意義は以下のようになる.

\begin{itemize}
\item 望ましい勤労観,職業観の育成
\item 学ぶこと,働くことの意義の理解,及びその関連性の把握
\item 啓発的経験と進路意識の伸長
\item 職業生活,社会生活に必要な知識,技術・技能の習得への理解や関心
\item 社会の構成員として共に生きる心を養い,社会奉仕の精神の涵養 等
\end{itemize}

\subsection{職場体験の効果的な展開}

職場体験では,ただ体験するだけで終わってしまい,本来の教育的機能を十分に発揮できていない傾向もある.

職場体験のポイントとしては以下が挙げられる.

\renewcommand{\arraystretch}{1.8}
\begin{table}[htbp]
    \centering
    \begin{tabularx}{\textwidth}{|X|X|X|}
        \hline
        ねらいの設定 & 実施計画の立案 & 体験先,保護者との連携 \\
        \hline
        事前指導の充実 & 実施期間中の指導体制 & 事後指導の充実 \\
        \hline
        評価 & & \\
        \hline
    \end{tabularx}
\end{table}

\section{キャリア形成に関する自己評価の意義}

キャリア教育に関する学習活動の遭遇や成果などの記録・作品を計画的に集積し,それらの学習成果物を積極的に活用することで,子どもが自己理解を深め,自らの将来について考える機会をつくることが大切である.

このときに集積しておくべき学習成果物としては,以下のようなものが挙げられる.

\begin{itemize}
\item 自己の将来や生き方に関する考えの記述
\item 自分の強み・弱みを書き出したワークシートなどの主観的な評価の記録
\item 友達や保護者,職場の人々による他者評価の記録
\item 職業レディネス・テスト,一般職業適性検査,YG性格検査などの客観的な評価の記録
\item 子どもが作成したレポート,ノート,作文,絵などの制作物
\item 教師による行動観察記録(子どもの発表や話し合いの様子など)
\end{itemize}

\section{キャリア・カウンセリングとは}

\begin{description}
    \item[キャリア・カウンセリングの定義] 「発達過程にある一人一人の子供たちが,個人差や特徴を生かして,学校生活における様々な体験を前向きに受け止め,日々の生活で直面する課題や問題を積極的・建設的に解決していくことを通じて,問題対処の力や態度を発達させ,自立的に生きていけるように支援することを目指す」ものである(文部科学省,2023).
\end{description}

\subsection{キャリア・カウンセリングのプロセス}

\begin{enumerate}
\item \textbf{現状(問題)の確認}

子どもの現状や不安を把握し,受容するとともにキャリア(進路・職業など)に対する意識やこれまでの取組みなどを把握する.

\item \textbf{子ども自身の自己理解やキャリアに対する理解を促す}

子どもの自己理解を深め,新たな可能性を発見させるためカウンセリングや心理検査などを行う.また,キャリアルートや求められる能力などキャリアに必要な情報の収集,理解を促すかかわりをする.

\item \textbf{目標や計画などに関する自己選択と自己決定を促す}

キャリアに関する目標や計画などを話し合いながら,子どもの自己選択や自己決定を促す.

\item \textbf{振り返りと次回についての確認}

面談の内容を振り返るとともに,次の面接の内容や時期,いつまでに何をするのかを決めるなど,今後のことを確認する.
\end{enumerate}

\section{体験活動とは}

\begin{description}
    \item[キャリア教育の実施] 「直接的体験を織り返す体験活動は,学びへの好奇心,課題発見等の学習への動機付けや意欲を高め,思考や実践,課題解決等の創意を広げ,次への体験や学びへの深化を促す『学びの過程』の基盤と成り得るもの」として重点が置かれている(国立教育政策研究所,2008).
\end{description}

キャリアは「選ぶ」だけではなく,選んだものを「つくりあげていく」という視点をもつことが大事で
あり,(1)主体的に進路を選択できるように「選ぶ」ための進路選択のレディネスを高め・育成する,
(2)選んだ進路をさらにつくりあげていく・育てていく,といった2軸が必要である.

このようなガイダンスの取組みの中で培われた力が,
選択後の過程でも,自分の役割の価値などに気づき,取捨選択する能力をもって
自分のキャリアをさらに積み重ねていく生き方を支えることにもつながっていくのである.

\begin{ascolorbox17}{質問}
\begin{enumerate}
\item 職場体験での失敗談はありますか?
\item ポートフォリオはしっかりと書いていましたか?
\item カウンセリングを受けて感じたことはありますか?
\end{enumerate}
\end{ascolorbox17}

\end{document}